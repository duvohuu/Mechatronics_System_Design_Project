\chapter{THIẾT KẾ GIẢI THUẬT ĐIỀU KHIỂN}
     \section{Giải thuật bám line}
          \hspace*{0.6cm}Bộ điều khiển giúp xe bám line bằng cách giảm thiểu sai số $e$ nhỏ nhất có thể thông qua điều khiển tốc độ 2 động cơ DC. Bộ điều khiển PID được lựa chọn cho bài toán bám line.
          \newline
          \hspace*{0.6cm}Xác định đầu vào và đầu ra của bộ điều khiển
          \begin{itemize}
               \item Input: sai số giữa điểm bám line $C$ và điểm tham chiếu $R$ trên đường line.
               \item Output: Tốc độ góc của hai bánh xe.
          \end{itemize}
          \hspace*{0.6cm}Từ chương 5 mô hình hóa ta có
          \begin{align}
               \begin{bmatrix}
                    e_1 \\
                    e_2 \\
                    e_3
                    \end{bmatrix} &= \begin{bmatrix}
                    \cos\varphi & \sin \varphi & 0 \\
                    -\sin\varphi & \cos \varphi & 0 \\
                    0 & 0 & 1
                    \end{bmatrix} \begin{bmatrix}
                    x_R - x_C \\
                    y_R - y_C \\
                    \varphi_R - \varphi_C
               \end{bmatrix} 
               \label{c6_e1}
          \end{align}
          \begin{align}
               \begin{bmatrix}
                    \dot{e_1} \\
                    \dot{e_2} \\
                    \dot{e_3}
                    \end{bmatrix} &= \begin{bmatrix}
                    v_R \cos e_3 \\
                    v_R \sin e_3 \\
                    \omega_R
                    \end{bmatrix} + \begin{bmatrix}
                    -1 & e_2 \\
                    0 & -d - e_1 \\
                    0 & -1
                    \end{bmatrix} \begin{bmatrix}
                    v_I \\
                    \omega_I
               \end{bmatrix}  
               \label{c6_e2}             
          \end{align}
          \hspace*{0.6cm}Hệ được mô tả bởi không gian trạng thái
          \begin{equation}
               \dot{x} = Ax + Bu 
               \label{c6_e3}
          \end{equation}
          \hspace*{0.6cm}Trong đó
          \begin{equation*}
               \dot{x} = \begin{bmatrix}
                    \dot{e_1} \\
                    \dot{e_2} \\
                    \dot{e_3}
               \end{bmatrix};
               x = \begin{bmatrix}
                    e_1 \\
                    e_2 \\
                    e_3
               \end{bmatrix}; 
               u = \begin{bmatrix}
                    v \\ 
                    \omega
               \end{bmatrix}
          \end{equation*}
          \hspace{0.6cm}Nhận thấy điểm cân bằng của hệ phi tuyến là $X_0 = [e_{10} \, e_{20} \, e_{30}] = [0 \, 0 \, 0]^{\text{T}}$. Phân tích và tuyến tính hóa hệ quanh điểm cân bằng ta tính được các ma trận
          \begin{equation*}
               A = \begin{bmatrix}
               0 & \omega & -v_R \times \sin e_3 \\
               -\omega & 0 & v_R \times \cos e_3 \\
               0 & 0 & 0
               \end{bmatrix}; \quad B = \begin{bmatrix}
               -1 & e_2 \\
               0 & -d - e_1 \\
               0 & -1
               \end{bmatrix}
          \end{equation*}
          Suy ra 
          \begin{align}
               \begin{bmatrix}
               \dot{e}_1 \\
               \dot{e}_2 \\
               \dot{e}_3
               \end{bmatrix} &= \begin{bmatrix}
               0 & \omega & -v_R \times \sin e_3 \\
               -\omega & 0 & v_R \times \cos e_3 \\
               0 & 0 & 0
               \end{bmatrix} \begin{bmatrix}
               e_1 \\
               e_2 \\
               e_3
               \end{bmatrix} + \begin{bmatrix}
               -1 & e_2 \\
               0 & -d - e_1 \\
               0 & -1
               \end{bmatrix} \begin{bmatrix}
               v \\
               \omega
               \end{bmatrix}\\
               &\approx \begin{bmatrix}
               0 & \omega & 0 \\
               -\omega & 0 & v_R \\
               0 & 0 & 0
               \end{bmatrix} \begin{bmatrix}
               e_1 \\
               e_2 \\
               e_3
               \end{bmatrix} + \begin{bmatrix}
               -1 & 0 \\
               0 & -d \\
               0 & -1
               \end{bmatrix} \begin{bmatrix}
               v \\
               \omega
               \end{bmatrix}
               \label{c6_e4}
          \end{align}
          

